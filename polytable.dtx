% \iffalse
%<*driver>
\ProvidesFile{polytable.drv}
\documentclass{ltxdoc}
\usepackage[T1]{fontenc}
\usepackage[utf8]{inputenc}
\usepackage{lmodern}
\usepackage[info]{polytable}
\begin{document}
   \DocInput{polytable.dtx}
\end{document}
%</driver>
%
% Copyright (C) 2003--2020 by Andres Loeh
%
% \fi
%
% \ProvidesFile{polytable.dtx}
%   [2020/07/14 v0.8.6 `polytable' package (Andres Loeh)]
% \GetFileInfo{polytable.dtx}
%
% \title{The \textsf{polytable} package}
% \author{Andres L\"oh\\ \texttt{polytable@andres-loeh.de}}
% \date{\filedate}
%
% \maketitle
%
% \begin{abstract}
% This package implements a variant of tabular/tabbing-like environments
% where columns can be given a name and entries can flexibly
% be placed between arbitrary columns. Complex alignment-based
% layouts, for example for program code, are possible.
% \end{abstract}
%
% \changes{v0.1}{2002/11/17}{Works.}
% \changes{v0.2}{2003/04/03}{Added pboxed. Added debug option.}
% \changes{v0.3}{2003/05/26}{Greatly enhanced pboxed.}
% \changes{v0.4}{2003/06/11}{Optional argument for ptable and parray.}
% \changes{v0.4.1}{2003/07/13}{Saving and restoring enhancements.}
% \changes{v0.4.2}{2003/07/16}{Fixed bug that caused infinite reruns.}
% \changes{v0.4.3}{2003/07/20}{Added some missing relaxes.}
% \changes{v0.5}{2003/07/22}{Documentation improved. To be put on CTAN.}
% \changes{v0.6}{2003/07/24}{Depends on lazylist instead of lambda now.}
% \changes{v0.7}{2003/11/20}{Check for undefined columns. Save counters.}
% \changes{v0.7.1}{2003/12/26}{Counters have not been restored correctly.}
% \changes{v0.7.2}{2004/02/27}{Fix for conflict with amsmath package.}
% \changes{v0.7.3}{2004/06/21}{Balance last line in pboxed environment.}
% \changes{v0.8}{2004/06/24}{New environments pmboxed, ptboxed.
%                            Removed original ptabular and parray.
%                            Interface overhaul.}
% \changes{v0.8.1}{2004/11/11}{Fixed some problems with the new
%                              interface, added the external (disk) queue.}
% \changes{v0.8.2}{2005/04/26}{New interface command.
%                              Minor beautifications.}
% \changes{v0.8.3}{2009/04/28}{Make sorting algorithm stable.}
% \changes{v0.8.4}{2009/11/01}{Make ptb output more stable.}
% \changes{v0.8.5}{2013/07/18}{Work around lazylist name clash
%                              with unicode-math.}
% \changes{v0.8.6}{2020/07/14}{Fix bug with column command redefinitions.}
%
% \section{Introduction}
%
% This package implements a variant of tabular-like environments.
% We will call these environments the |poly|-environments to
% distinguish them from the standard ones as provided by the
% \LaTeX\ kernel or the \textsf{array} package.
%
% Other than in standard tables, each column has a name. For
% instance, the commands\\
% |\column{foo}{l}|\\
% |\column{bar}{r}|\\
% -- when used within a |poly|-environment -- define a column
% with name |foo| that is left-aligned, and a column with name
% |bar| that is right-aligned.
%
% Once a couple of columns have been defined, the text is specified
% in a series of |\fromto| commands. Instead of specifying text
% per column in order, separating columns with |&|, we give the
% name of the column where the content should start, and the name
% of the column before which the content should stop.
% To typeset the text ``I'm aligned!'' in the column |foo|,
% we could thus use the command\\
% |\fromto{foo}{bar}{I'm aligned}|\\
% Several |\fromto|-commands can be used to typeset a complete
% line of the table. A new line can be started with |\\|.
%
% The strength of this approach is that it implicitly handles
% cases where different lines have different alignment properties.
% Not all column names have to occur in all lines.
%
% \section{A complete example}
%
% Figure~\ref{tab:example}
% is an example that is designed to show the
% capabilities of this package. In particular, it is \emph{not}
% supposed to look beautiful.
%
% \begin{figure}[htb]
% \centering
% \begin{ptboxed}
% \defaultcolumn{l|}\column{.}{|l|}
% \> left
% \=3 first of three \> second of three \> third of three
% \=r right \\
% \defaultcolumn{r|}\> left \=2 middle 1/2 \> middle 2/2 \=r right \\
% \> left \=3 middle 1/3 \> middle 2/3 \> middle 3/3 \=r right \\
% \> left
% \=2 first of two middle columns \> second of two middle columns
% \=r right \\
% \end{ptboxed}
% \caption{Example table}%
% \label{tab:example}
% \end{figure}
%
% The example table consists of four lines. All lines have
% some text on the left and on the right, but the middle part
% follows two different patterns: the first and the third line
% have three middle columnss that should be aligned, the second
% and the fourth line have two (right-aligned) middle columns
% that should be aligned, but otherwise independent of the three
% middle columns in the other lines.
%
% Vertical bars are used to clarify where one column ends and
% the next column starts in a particular line. Note that the first
% and the third line are completely aligned. Likewise, the
% second and the fourth line are. However, the fact that the
% bar after the text ``middle 1/2'' ends up between the two
% bars delimiting the column with ``second of three'' in it
% is just determined by the length of the text ``first of two
% middle columns'' in the last line. This text fragment
% is wider than the first of the three middle columns, but not
% wider than the first two of the three middle columns.
%
% Let's have a look at the input for the example table:
% \begin{verbatim}
% \begin{ptboxed}
% \defaultcolumn{l|}\column{.}{|l|}
% \> left
% \=3 first of three \> second of three \> third of three
% \=r right \\
% \defaultcolumn{r|}\> left \=2 middle 1/2 \> middle 2/2 \=r right \\
% \> left \=3 middle 1/3 \> middle 2/3 \> middle 3/3 \=r right \\
% \> left
% \=2 first of two middle columns \> second of two middle columns
% \=r right \\
% \end{ptboxed}
% \end{verbatim}
% First, columns are declared, including the vertical lines.
% Note that there is a final column |end| being declared
% that is only used as the end column in the |\fromto| statements.
% A future version of this package
% will probably get rid of the need to define
% such a column.
% After the column definitions, the lines are typeset by a
% series of |\fromto| commands, separated by |\\|.
% Note that the first and third column do not use |m12|, |m22|.
% Similarly, the second and fourth column do not use
% |m13|, |m23|, and |m33|.
%
% So far, one could achieve the same with an ordinary |tabular|
% environment. The table would have 6 columns. One left and right,
% the other four for the middle: the first and third line would
% use the first of the four columns, then place the second entry
% in a |\multicolumn| of length 2, and then use the fourth column
% for the third entry. Likewise, the other lines would place both
% their entries in a |\multicolumn| of length 2. In fact, this
% procedure is very similar to the way the |ptabular| environment
% is implemented.
%
% The problem is, though, that we need the information that
% the first of the two middle columns ends somewhere in the middle
% of the second of the three columns, as observed above. If we
% slightly modify the texts to be displayed in the middle columns,
% this situation changes. Figure~\ref{tab:variation} shows
% two variants of the example table. The input is the same,
% only that the texts contained in some columns have slightly
% changed. As you can see, the separator between the first and
% second middle column in the second and fourth lines of the
% tables now once ends up within the first, once within the
% third of the three middle columns of the other lines.
%
% \begin{figure}[htb]
% \centering
% \begin{ptboxed}
% \column{left}{|l|}
% \column{right}{l|}
% \column{m13}{l|}
% \column{m23}{l|}
% \column{m33}{l|}
% \column{m12}{r|}
% \column{m22}{r|}
% \column{end}{l}
% \fromto{left}{m13}{left}
% \fromto{m13}{m23}{first of three}
% \fromto{m23}{m33}{second of three}
% \fromto{m33}{right}{third of three}
% \fromto{right}{end}{right}
% \\
% \fromto{left}{m12}{left}
% \fromto{m12}{m22}{middle 1/2}
% \fromto{m22}{right}{middle 2/2}
% \fromto{right}{end}{right}
% \\
% \fromto{left}{m13}{left}
% \fromto{m13}{m23}{middle 1/3}
% \fromto{m23}{m33}{middle 2/3}
% \fromto{m33}{right}{middle 3/3}
% \fromto{right}{end}{right}
% \\
% \fromto{left}{m12}{left}
% \fromto{m12}{m22}{first of two}
% \fromto{m22}{right}{second of two}
% \fromto{right}{end}{right}
% \end{ptboxed}
% \par\bigskip
% \begin{ptboxed}
% \column{left}{|l|}
% \column{right}{l|}
% \column{m13}{l|}
% \column{m23}{l|}
% \column{m33}{l|}
% \column{m12}{r|}
% \column{m22}{r|}
% \column{end}{l}
% \fromto{left}{m13}{left}
% \fromto{m13}{m23}{first of three}
% \fromto{m23}{m33}{second of three}
% \fromto{m33}{right}{third of three}
% \fromto{right}{end}{right}
% \\
% \fromto{left}{m12}{left}
% \fromto{m12}{m22}{middle 1/2}
% \fromto{m22}{right}{middle 2/2}
% \fromto{right}{end}{right}
% \\
% \fromto{left}{m13}{left}
% \fromto{m13}{m23}{middle 1/3}
% \fromto{m23}{m33}{middle 2/3}
% \fromto{m33}{right}{middle 3/3}
% \fromto{right}{end}{right}
% \\
% \fromto{left}{m12}{left}
% \fromto{m12}{m22}{the first of two middle columns}
% \fromto{m22}{right}{2/2}
% \fromto{right}{end}{right}
% \end{ptboxed}
% \caption{Variants of the example table}%
% \label{tab:variation}
% \end{figure}
%
% If one wants the general case using the |\multicolumn|
% approach, one thus has to measure the widths of the entries
% of the columns to compute their relative position. In essence,
% this is what the package does for you.
%
% \section{Haskell code example}
%
% I have written this package mainly for one purpose: to be able
% to beautifully align Haskell source code. Haskell is a functional
% programming language where definitions are often grouped into
% several declarations. I've seen programmers exhibit symmetric
% structures in different lines by adding spaces in their source
% code files in such a way that corresponding parts in different
% definitions line up. On the other hand, as Haskell allows
% user-defined infix operators, some programmers like their symbols
% to be typeset as \LaTeX\ symbols, not as typewriter code.
% But using \LaTeX\ symbols and a beautiful proportional font
% usually destroys the carefully crafted layout and alignment.
%
% With lhs2\TeX, there is now a preprocessor available that preserves
% the source code's internal alignment by mapping the output onto
% \textsf{polytable}'s environments.
% Figure~\ref{tab:haskell} is an example of how
% the output of lhs2\TeX\ might look like.
%
% \newcommand{\HsVar}[1]{\mathop{\mathit{\strut #1}}}
% \newcommand{\HsCon}[1]{\mathop{\mathsf{\strut #1}}}
% \newcommand{\HsKey}[1]{\mathop{\mathbf{\strut #1}}}
% \newcommand{\map}{\HsVar{map}}
% \newcommand{\round}{\HsVar{round}}
% \newcommand{\TA}{\HsVar{a}}
% \newcommand{\TB}{\HsVar{b}}
% \newcommand{\TF}{\HsVar{f}}
% \newcommand{\TX}{\HsVar{x}}
% \newcommand{\TY}{\HsVar{y}}
% \newcommand{\TN}{\HsVar{n}}
% \newcommand{\TM}{\HsVar{m}}
% \newcommand{\TR}{\HsVar{r}}
% \newcommand{\TXS}{\HsVar{xs}}
% \newcommand{\RealFrac}{\mathop{\mathsf{RealFrac}}}
% \newcommand{\Integral}{\mathop{\mathsf{Integral}}}
% \newcommand{\LET}{\mathop{\mathbf{let}}}
% \newcommand{\IN}{\mathop{\mathbf{in}}}
% \newcommand{\IF}{\mathop{\mathbf{if}}}
% \newcommand{\THEN}{\mathop{\mathbf{then}}}
% \newcommand{\ELSE}{\mathop{\mathbf{else}}}
% \newcommand{\OF}{\mathop{\mathbf{of}}}
% \newcommand{\CASE}{\mathop{\mathbf{case}}}
% \newcommand{\CLASS}{\mathop{\mathbf{class}}}
% \newcommand{\WHERE}{\mathop{\mathbf{where}}}
% \newcommand{\Eq}{\mathop{\mathsf{Eq}}}
% \newcommand{\Ord}{\mathop{\mathsf{Ord}}}
%
% \setlength{\arraycolsep}{.5ex}
%
% \begin{figure}
% \[
% \begin{pmboxed}
% \column{left}{l}
% \column{body}{l}
% \column{end}{l}
% \column{rel}{c}
% \column{rhs}{l}
% \>[left] \CLASS\ (\Eq\TA)\Rightarrow\Ord\TA\WHERE \\
% \>[left] \qquad\
% \>[body] \HsVar{compare}
% \>[rel]  ::
% \>[rhs]  \TA\rightarrow\TA\rightarrow\HsCon{Ordering} \\
% \>[left] \qquad\
% \>[body] (<),(\leq),(\geq),(>)
% \>[rel]  ::
% \>[rhs]  \TA\rightarrow\TA\rightarrow\HsCon{Bool}
% \\
% \>[left] \qquad\
% \>[body] \HsVar{max},\HsVar{min}
% \>[rel]  ::
% \>[rhs]  \TA\rightarrow\TA\rightarrow\HsCon{Bool}
% \\
% \\
% \>[body] \mbox{--- Minimal complete definition:
%                  $(\leq)$ or $\HsVar{compare}$}
% \\
% \>[body] \mbox{--- using $\HsVar{compare}$ can be more efficient
%                  for complex types}
% \\
% \column{cgrd}{l}
% \>[body] \HsVar{compare}\TX\TY
% \>[cgrd] \vert\ \TX\equiv\TY
% \>[rel]  =
% \>[rhs]  \HsCon{EQ}
% \\
% \>[cgrd] \vert\ \TX\leq\TY
% \>[rel]  =
% \>[rhs]  \HsCon{LT}
% \\
% \>[cgrd] \vert\ \HsVar{otherwise}
% \>[rel]  =
% \>[rhs]  \HsCon{GT}
% \\
% \>[body] \TX\leq\TY
% \>[rel]  =
% \>[rhs]  \HsVar{compare}\TX\TY\not\equiv\HsCon{GT}
% \\
% \>[body] \TX < \TY
% \>[rel]  =
% \>[rhs]  \HsVar{compare}\TX\TY\equiv\HsCon{LT}
% \\
% \>[body] \TX\geq\TY
% \>[rel]  =
% \>[rhs]  \HsVar{compare}\TX\TY\not\equiv\HsCon{LT}
% \\
% \>[body] \TX > \TY
% \>[rel]  =
% \>[rhs]  \HsVar{compare}\TX\TY\equiv\HsCon{GT}
% \\
% \>[body] \HsVar{max}\TX\TY
% \>[cgrd] \vert\ \TX\leq\TY
% \>[rel]  =
% \>[rhs]  \TY
% \\
% \>[cgrd] \vert\ \HsVar{otherwise}
% \>[rel]  =
% \>[rhs]  \TX
% \\
% \>[body] \HsVar{min}\TX\TY
% \>[cgrd] \vert\ \TX\leq\TY
% \>[rel]  =
% \>[rhs]  \TX
% \\
% \>[cgrd] \vert\ \HsVar{otherwise}
% \>[rel]  =
% \>[rhs]  \TY
% \end{pmboxed}
% \]
% \caption{Haskell code example}
% \label{tab:haskell}
% \end{figure}
%
% Of course, this could be useful for other programming languages
% as well. In fact, lhs2\TeX\ can be tweaked to process several
% experimental languages that are based on Haskell, but I can imagine
% that this package could generally prove useful to typeset program
% code.
%
% \section{Other applications}
%
% Although I have written this package for a specific purpose, I
% am very much interested to hear of other potential application
% areas. Please tell me if you found a use for this package and
% do not hesitate to ask for additional features that could convince
% you to use the package for something.
%
% \section{The \textsf{lazylist} package}
%
% Internally, this package makes use of Alan Jeffrey's excellent
% \textsf{lazylist} package, which provides an implementation of
% the lambda calculus using fully expandable control sequences.
% Unfortunately, |lazylist.sty| is not included in most common
% \TeX\ distributions, so you might need to fetch it from CTAN
% separately.
%
% \section{Reference}
%
% \subsection{Setup}
%
% New in v0.8: We allow to implicitly define columns by just
% using the names during table content specification, without
% having declared them formally using |\column| (see below).
%
% \DescribeMacro{\nodefaultcolumn}
% By default, though, this behaviour is turned off, because the
% use of a misspelled column is often an error. Thus, by default
% or after the command |\nodefaultcolumn| is used, all columns
% must be declared.
%
% \DescribeMacro{\defaultcolumn}
% Using a statement of the form
% |\defaultcolumn{|\meta{spec}|}|, implicit columns can
% be activated. All undefined columns that are encountered
% will be assumed to have format string \meta{spec}.
%
% \subsection{The environments}
%
% \DescribeEnv{pboxed}
% \DescribeEnv{ptboxed}
% \DescribeEnv{pmboxed}
% \DescribeEnv{ptabular}
% \DescribeEnv{parray}
% New in v0.8: There are now five environments that this package
% provides: in addition to the former |ptabular|, |parray|, and
% |pboxed|, there is now also |ptboxed| and |pmboxed|.
% The environment |pboxed|
% typesets the material in boxes of the calculated length, but in
% normal paragraph mode. The advantage is that
% there can be page breaks within the table.
% Note that you should start a new (probably nonindented)
% paragraph before beginning a |pboxed|. All lines that a |pboxed|
% produces are of equal length, so it should be possible to center
% or right-align the material.

% Both |ptboxed| and |pbmoxed| are like |pboxed|, but
% pre-wrapped in a |tabular| or |array| environment,
% respectively, and
% thus not page-breakable but less fragile (or should I just
% say: easier to use) than the unwrapped |pboxed|.
% With those, there
% is no need for |ptabular| and |parray| anymore --
% which were directly based on (and translated into) usual
% |tabular| and |array| environments as provided by the |array|
% package. The environments are still supported, to ensure
% compatibility, but they are mapped to |ptboxed| and |pmboxed|,
% respectively.
%
% The |pmboxed| and |parray| environments assume math mode, whereas
% |ptboxed| and |ptabular| assume text mode. The |pboxed| environment
% works in both text and math modes and adapts to whatever is
% currently active.
%
% I wrote in previous versions that one additional environment, namely
% |plongtable|, a poly-version of the |longtable| environment, was planned.
% Page-breaking of |pboxed| works fine, and I do not see a
% real need for a |plongtable| environment anymore. If you would like
% it anyway, feel free to mail me and inform me about the advantages
% it would have over |pboxed|.
%
% The interface is the same for all of the environments.
%
% \subsection{The interface}
%
% \begin{center}
% \textbf{(Note: this has changed significantly in v0.8!)}
% \end{center}
%
% In each of the environments, the following commands can be
% used. All other commands should be used with care if placed
% directly inside such an environment: the contents of a polytable
% are processed multiple times; hence, if your commands generate
% output, the output will end up multiple times in the final
% document, and destroy your layout.
%
% \DescribeMacro{\column}
% With |\column[|\meta{dimen}|]{|\meta{columnid}|}{|\meta{spec}|}|,
% a new column
% \meta{columnid} is specified. The name of the column can be
% any sequence of alphanumerical characters. The \meta{spec} is
% a format string for that particular column, and it can contain
% the same constructs that can be used in format strings of normal
% tables or arrays.
% However, it must only contain the description for \emph{one}
% column.
%
% As long as the save/restore feature (explained below) is not used,
% |\column| definitions are local to one table.
% One can define a column multiple times within one table:
% a warning will be produced,
% and the second format string will be used for the complete table.
%
% Columns must be defined before use when implicit columns using
% |\defaultcolumn| are disabled.
%
% A minimal horizontal position \meta{dimen} (relative to the
% beginning of the table) can be specified for a column, which
% defaults to 0pt.
%
% \DescribeMacro{\=}
% The command |\={|\meta{fromid}|}[|\meta{spec}|]| instructs the
% package to prepare for a new entry starting at column \meta{fromid}.
% Everything that follows the command, up to the next
% interface-command (except |\column|) or the end of the environment
% is interpreted as contents of the entry and can be arbitrary
% \LaTeX\ code that has balanced grouping. The column specifier
% \meta{spec} can be used to define a different formatting for this
% particular entry. If the specifier starts with an exclamation mark
% (|!|), it will be used as specifier for all entries of that column.
% The use of multiple exclamation-mark specifiers for the same
% column name gives precedence to the last one, but you should not
% rely on this behaviour as it may change in the future.
%
% Note that, contrary to normal \LaTeX\ behaviour, the second argument
% is the optional argument. Therefore, if an entry starts with an
% opening bracket |([)|,
% and the optional argument is omitted, a |\relax| should be inserted
% between command and opening bracket, as otherwise the entry would
% be treated as the optional argument.
%
% \DescribeMacro{\>}
% The command |\>[|\meta{fromid}|][|\meta{spec}|]| is a variant of
% |\=| where both arguments are optional. If no column name is given
% then the current column name, postfixed by a dot (|.|), is assumed
% as column name. At the beginning of a line, a single dot (|.|)
% will be assumed as a name. The \meta{spec} argument has the same
% behaviour as for |\=|. Note that if the specifier is given, the
% column name must be given as well, but may be left empty if the
% default choice is desired. For instance, |\>[][r]|, will move to
% the next column, which will be typeset right-aligned.
%
% \DescribeMacro{\<}
% The command |\<[|\meta{fromid}|]| ends the current entry at the
% boundary specified by \meta{fromid}. This macro can be used instead
% of |\>| or |\=| if an entry should be ended without immediately
% starting a new one. Differences in behaviour occur if \meta{fromid}
% is associated with a non-trivial column format string.
% TODO: Improve this explanation.
%
% \DescribeMacro{\fromto}
% The call |\fromto{|\meta{fromid}|}{|\meta{toid}|}{|\meta{text}|}|
% will typeset \meta{text} in the current line, starting at column
% \meta{fromid} and ending before column \meta{toid}, using the
% format string specified for \meta{fromid}.
%
% A line of a table usually consists of multiple |\fromto| statements.
% Each statement's starting column should be either the same as the
% end column of the previous statement, or it will be assumed that
% the start column is located somewhere to the right of the previous
% end column. The user is responsible to not introduce cycles in the
% (partial) order of columns. If such a cycle is specified, the
% current algorithm will loop, causing a |dimension too large| error
% ultimately. TODO: catch this error.
%
% \DescribeMacro{\\}
% \DescribeMacro{\nextline}
% The command |\\| (or, now deprecated, |\nextline|) ends one line
% and begins the next.
% There is no need to end the last line. One can pass an optional
% argument, as in |\\[|\meta{dimen}|]|, that will add
% \meta{dimen} extra space between the lines.
%
% \subsection{A warning}
%
% The contents of the table are processed multiple times because
% the widths of the entries are measured. Global assignments that
% modify registers and similar things can thus result in unexpected
% behaviour.
% New in v0.7: \LaTeX\ counters (i.e., counters defined by
% |\newcounter|) are protected now. They will be reset after
% each of the trial runs.
%
% \subsection{Saving column width information}
%
% Sometimes, one might want to reuse not only the same column, but
% exactly the same alignment as in a previous table. An example
% would be a fragment of program code, which has been broken into
% several pieces, with documentation paragraphs added in between.
%
% \DescribeMacro{\savecolumns}
% \DescribeMacro{\restorecolumns}
% With |\savecolumns[|\meta{setid}|]|, one can save the information
% of the current table for later reuse. The name |setid| can be an
% arbitrary sequence of alphanumeric characters. It does \emph{not} share
% the same namespace as the column names. The argument is optional;
% if it is omitted, a default name is assumed. Later, one can restore
% the information (multiple times, if needed) in other tables, by
% issuing a |\restorecolumns[|\meta{setid}|]|.
%
% This feature requires to pass information backwards in the general
% case, as column widths in later environments using one specific
% column set might influence the layout of earlier environments.
% Therefore, information is written into the |.aux| file, and
% sometimes, a warning is given that a rerun is needed. Multiple
% reruns might be required to get all the widths right.
%
% I have tried very hard to avoid producing rerun warnings infinitely
% except if there are really cyclic dependencies between columns.
% Still, if it happens or something seems to be broken, it often
% is a good idea to remove the |.aux| file and start over. Be sure
% to report it as a bug, though.
%
% Figure~\ref{tab:saverestore} is an example of the Haskell code
% example with several comments inserted. The source of this file
% shows how to typeset the example.
% \begin{figure}
% \setlength{\parindent}{0pt}%
% \setlength{\parskip}{1ex}%
% \hrule
% We introduce a new type class $\Ord$ for objects that admit
% an ordering. It is based on the $\Eq$ class:
%
% \(
% \begin{pboxed}
% \savecolumns
% \column{left}{>{\quad}l}
% \column{end}{l}
% \fromto{left}{end}{\CLASS\ (\Eq\TA)\Rightarrow\Ord\TA\WHERE}
% \end{pboxed}
% \)
%
% The next three lines give the type signatures for all
% the methods of the class.
%
% \(
% \begin{pboxed}
% \restorecolumns
% \column{body}{l}
% \column{rel}{c}
% \column{rhs}{l}
% \fromto{left}{body}{\qquad\ }
% \fromto{body}{rel}{\HsVar{compare}}
% \fromto{rel}{rhs}{::}
% \fromto{rhs}{end}{\TA\rightarrow\TA\rightarrow\HsCon{Ordering}}
% \\
% \fromto{left}{body}{\qquad\ }
% \fromto{body}{rel}{(<),(\leq),(\geq),(>)}
% \fromto{rel}{rhs}{::}
% \fromto{rhs}{end}{\TA\rightarrow\TA\rightarrow\HsCon{Bool}}
% \\
% \fromto{left}{body}{\qquad\ }
% \fromto{body}{rel}{\HsVar{max},\HsVar{min}}
% \fromto{rel}{rhs}{::}
% \fromto{rhs}{end}{\TA\rightarrow\TA\rightarrow\HsCon{Bool}}
% \\[1ex]
% \fromto{body}{end}{\mbox{--- Minimal complete definition:
%   $(\leq)$ or $\HsVar{compare}$}}
% \\
% \fromto{body}{end}{\mbox{--- using $\HsVar{compare}$ can be more efficient
%   for complex types}}
% \end{pboxed}
% \)
%
% As the comment above says, it is sufficient to define either
% $(\leq)$ or $\HsVar{compare}$ to get a complete instance. All
% of the class methods have default definitions. First, we can
% define $\HsVar{compare}$ in terms of $(\leq)$. The result type
% of $\HsVar{compare}$ is an $\HsCon{Ordering}$, a type consisting
% of only three values: $\HsCon{EQ}$ for ``equality'',
% $\HsCon{LT}$ for ``less than'', and $\HsCon{GT}$ for ``greater
% than''.
%
% \(
% \begin{pboxed}
% \restorecolumns
% \column{cgrd}{l}
% \fromto{body}{cgrd}{\HsVar{compare}\TX\TY}
% \fromto{cgrd}{rel}{\vert\ \TX\equiv\TY}
% \fromto{rel}{rhs}{=}
% \fromto{rhs}{end}{\HsCon{EQ}}
% \\
% \fromto{cgrd}{rel}{\vert\ \TX\leq\TY}
% \fromto{rel}{rhs}{=}
% \fromto{rhs}{end}{\HsCon{LT}}
% \\
% \fromto{cgrd}{rel}{\vert\ \HsVar{otherwise}}
% \fromto{rel}{rhs}{=}
% \fromto{rhs}{end}{\HsCon{GT}}
% \end{pboxed}
% \)
%
% All the other comparison operators can be defined
% in terms of $\HsCon{compare}$:
%
% \(
% \begin{pboxed}
% \restorecolumns
% \fromto{body}{rel}{\TX\leq\TY}
% \fromto{rel}{rhs}{=}
% \fromto{rhs}{end}{\HsVar{compare}\TX\TY\not\equiv\HsCon{GT}}
% \\
% \fromto{body}{rel}{\TX < \TY}
% \fromto{rel}{rhs}{=}
% \fromto{rhs}{end}{\HsVar{compare}\TX\TY\equiv\HsCon{LT}}
% \\
% \fromto{body}{rel}{\TX\geq\TY}
% \fromto{rel}{rhs}{=}
% \fromto{rhs}{end}{\HsVar{compare}\TX\TY\not\equiv\HsCon{LT}}
% \\
% \fromto{body}{rel}{\TX > \TY}
% \fromto{rel}{rhs}{=}
% \fromto{rhs}{end}{\HsVar{compare}\TX\TY\equiv\HsCon{GT}}
% \end{pboxed}
% \)
%
% Finally, there are default definitions for
% $\HsVar{max}$ and $\HsVar{min}$ in terms of $(\leq)$.
%
% \(
% \begin{pboxed}
% \restorecolumns
% \fromto{body}{cgrd}{\HsVar{max}\TX\TY}
% \fromto{cgrd}{rel}{\vert\ \TX\leq\TY}
% \fromto{rel}{rhs}{=}
% \fromto{rhs}{end}{\TY}
% \\
% \fromto{cgrd}{rel}{\vert\ \HsVar{otherwise}}
% \fromto{rel}{rhs}{=}
% \fromto{rhs}{end}{\TX}
% \\
% \fromto{body}{cgrd}{\HsVar{min}\TX\TY}
% \fromto{cgrd}{rel}{\vert\ \TX\leq\TY}
% \fromto{rel}{rhs}{=}
% \fromto{rhs}{end}{\TX}
% \\
% \fromto{cgrd}{rel}{\vert\ \HsVar{otherwise}}
% \fromto{rel}{rhs}{=}
% \fromto{rhs}{end}{\TY}
% \end{pboxed}
% \)
% \hrule
% \caption{Commented Haskell code example}
% \label{tab:saverestore}
% \end{figure}
%
% \section{The Code}
%    \begin{macrocode}
%<*package>
\NeedsTeXFormat{LaTeX2e}
\ProvidesPackage{polytable}%
   [2020/07/14 v0.8.6 `polytable' package (Andres Loeh)]
%    \end{macrocode}
% New in v0.7.2:
% The \textsf{amsmath} package clashes with \textsf{lazylist}:
% both define the command |\And|. Although it would certainly
% be better to find another name in \textsf{lazylist}, we
% take precautions for now. (Note that this will still fail
% if \textsf{lazylist} is already loaded -- but then it's not
% our problem \dots
%    \begin{macrocode}
\let\PT@original@And\And
\let\PT@original@Not\Not
\RequirePackage{lazylist}
\let\PT@And\And
\let\PT@Not\Not
\def\PT@prelazylist
  {\let\And\PT@And
   \let\Not\PT@Not}
\def\PT@postlazylist
  {\let\And\PT@original@And
   \let\Not\PT@original@Not}
\PT@postlazylist
\RequirePackage{array}
%    \end{macrocode}
%
% The option debug will cause (a considerable amount of)
% debugging output to be printed. The option info is a
% compromise, printing some status information for each table,
% but no real debugging information. The option silent, on
% the other hand, will prevent certain warnings from being
% printed.
%    \begin{macrocode}
\DeclareOption{debug} {\AtEndOfPackage\PT@debug}
\DeclareOption{info}  {\AtEndOfPackage\PT@info}
\DeclareOption{silent}{\AtEndOfPackage\PT@silent}
%    \end{macrocode}
%
% First, we declare a couple of registers that we will need later.
%    \begin{macrocode}
\newdimen\PT@colwidth
\newcount\PT@cols
\newcount\PT@table
\newtoks\PT@toks
\newif\ifPT@changed
\newread\PT@in
\newwrite\PT@out
%    \end{macrocode}
% In |\PT@allcols|, we will store the list of all columns, as a
% list as provided by the \textsf{lazylist} package. We initialise
% it to the empty list, which is represented by |\Nil|. In v0.9,
% we will have a second list that only contains the public columns.
%    \begin{macrocode}
\def\PT@allcols{\Nil}
%\def\PT@allpubliccols{\Nil}
\let\PT@infromto\empty
%    \end{macrocode}
% These are flags and truth values. TODO: Reduce and simplify.
%    \begin{macrocode}
\let\PT@currentwidths\empty
\def\PT@false{0}
\def\PT@true{1}
\let\PT@inrestore\PT@false
%    \end{macrocode}
%
% \begin{macro}{\defaultcolumn}
% \begin{macro}{\nodefaultcolumn}
% \begin{macro}{\PT@defaultcolumnspec}
% The macro |\PT@defaultcolumnspec| contains, if defined, the
% current default specifier that is assumed for undefined columns.
% The other two commands can be used to set the specifier.
%    \begin{macrocode}
\newcommand{\defaultcolumn}[1]{\gdef\PT@defaultcolumnspec{#1}}
\newcommand{\nodefaultcolumn}{\global\let\PT@defaultcolumnspec\undefined}
\DeclareOption{defaultcolumns}{\defaultcolumn{l}}
%    \end{macrocode}
% \end{macro}
% \end{macro}
% \end{macro}
%
% \begin{macro}{\memorytables}
% \begin{macro}{\disktables}
% \begin{macro}{\PT@add}
% \begin{macro}{\PT@split}
% \begin{macro}{\PT@prearewrite}
% \begin{macro}{\PT@preareread}
% \begin{macro}{\PT@finalize}
% These macros steer where the end-column queue is stored, which
% can be either in memory or on disk. The default is on disk,
% because that's more reliable for large tables. There is a package
% option |memory| to make |\memorytables| the default.
%    \begin{macrocode}
\newcommand{\memorytables}{%
  \let\PT@preparewrite\@gobble
  \let\PT@add         \PT@addmem
  \let\PT@prepareread \PT@preparereadmem
  \let\PT@split       \PT@splitmem
  \let\PT@finalize    \relax
}
\newcommand{\disktables}{%
  \let\PT@preparewrite\PT@preparewritefile
  \let\PT@add         \PT@addfile
  \let\PT@prepareread \PT@preparereadfile
  \let\PT@split       \PT@splitfile
  \let\PT@finalize    \PT@finalizefile
}
\DeclareOption{memory}{\AtEndOfPackage\memorytables}
\ProcessOptions
%    \end{macrocode}
% \end{macro}
% \end{macro}
% \end{macro}
% \end{macro}
% \end{macro}
% \end{macro}
% \end{macro}
%
% \begin{macro}{\PT@debug}
% \begin{macro}{\PT@info}
% \begin{macro}{\PT@debug@}
% \begin{macro}{\PT@typeout@}
% \begin{macro}{\PT@silent}
% \begin{macro}{\PT@warning}
% Similar to the \textsf{tabularx} package, we add macros to
% print debugging information to the log. Depending on package
% options, we can set or unset them.
%    \begin{macrocode}
\newcommand*{\PT@debug}
  {\def\PT@debug@ ##1{\typeout{(polytable) ##1}}
   \PT@info}
\newcommand*{\PT@info}
  {\def\PT@typeout@ ##1{\typeout{(polytable) ##1}}}
\let\PT@debug@\@gobble
\let\PT@typeout@\@gobble
\def\PT@warning{\PackageWarning{polytable}}%
\def\PT@silent
  {\let\PT@typeout@\@gobble\let\PT@warning\@gobble}
%    \end{macrocode}
% \end{macro}
% \end{macro}
% \end{macro}
% \end{macro}
% \end{macro}
% \end{macro}
%
% \begin{macro}{\PT@aligndim}
% \begin{macro}{\PT@aligncol}
% The first is (almost) stolen from the \textsf{tabularx}-package, to nicely
% align dimensions in the log file. TODO: fix some issues.
% The other command is for column names.
%    \begin{macrocode}
\def\PT@aligndim#1#2#3\@@{%
  \ifnum#1=0
    \if #2p%
      \PT@aligndim@0.0pt\space\space\space\space\space\@@
    \else
      \PT@aligndim@#1#2#3\space\space\space\space\space\space\space\space\@@
    \fi
  \else
    \PT@aligndim@#1#2#3\space\space\space\space\space\space\space\space\@@
  \fi}

\def\PT@aligndim@#1.#2#3#4#5#6#7#8#9\@@{%
  \ifnum#1<10 \space\fi
  \ifnum#1<100 \space\fi
  \ifnum#1<\@m\space\fi
  \ifnum#1<\@M\space\fi
  #1.#2#3#4#5#6#7#8\space\space}

\def\PT@aligncol#1{%
  \PT@aligncol@#1\space\space\space\space\space\space\space\space\@@}

\def\PT@aligncol@#1#2#3#4#5#6#7#8#9\@@{%
  #1#2#3#4#5#6#7#8\space\space}
%    \end{macrocode}
% \end{macro}
% \end{macro}
%
% \begin{macro}{\PT@rerun}
% This macro can be called at a position where we know that
% we have to rerun \LaTeX\ to get the column widths right. It
% issues a warning at the end of the document.
%    \begin{macrocode}
\def\PT@rerun
  {\PT@typeout@{We have to rerun LaTeX ...}%
   \AtEndDocument
     {\PackageWarning{polytable}%
        {Column widths have changed. Rerun LaTeX.\@gobbletwo}}%
   \global\let\PT@rerun\relax}
%    \end{macrocode}
% \end{macro}
%
% \begin{macro}{\PT@currentcol}
% \begin{macro}{\PT@currentcolumn}
% Both macros are used during typesetting to store the current
% column. The differences are subtle. TODO: remove
% one of the two, if possible. The |\PT@currentcol| variant
% contains the internal name, whereas the |\PT@currentcolumn|
% variant contains the external name.
% \end{macro}
% \end{macro}
%
% \subsection{Macro definition tools}
%
% \begin{macro}{\PT@listopmacro}
% \begin{macro}{\PT@consmacro}
% \begin{macro}{\PT@appendmacro}
% This assumes that |#2| is a list macro and |#3| is a new list element.
% The macro |#2| should, after the call, expand to the list with the
% new element |#1|ed. Because we don't know the number of tokens in |#3|,
% we use a temporary macro |\PT@temp| (which is used frequently throughout
% the package).
%    \begin{macrocode}
\def\PT@listopmacro #1#2#3% #1 #3 to the list #2
  {\def\PT@temp{#1{#3}}%
   \expandafter\expandafter\expandafter
     \def\expandafter\expandafter\expandafter
     #2\expandafter\expandafter\expandafter
     {\expandafter\PT@temp\expandafter{#2}}}

\def\PT@consmacro{\PT@listopmacro\Cons}
\def\PT@appendmacro{\PT@listopmacro\Cat}
%    \end{macrocode}
% \end{macro}
% \end{macro}
% \end{macro}
% The follwing macro can be used to add something to
% the end of a control structure.
%    \begin{macrocode}
\def\PT@gaddendmacro #1#2% add #2 to the end of #1
  {\PT@expanded{\gdef #1}{#1#2}}
%    \end{macrocode}
% \begin{macro}{\PT@expanded}
% This macro expands its second argument once before passing
% it to the first argument. It is like |\expandafter|, but
% works on grouped arguments instead of tokens.
%    \begin{macrocode}
\def\PT@expanded #1#2%
  {\expandafter\Twiddle\expandafter\Identity\expandafter{#2}{#1}}
%    \end{macrocode}
% \end{macro}
% \begin{macro}{\PT@enamedef}
% This is much like |\@namedef|, but it expands |#2| once.
%    \begin{macrocode}
\def\PT@enamedef #1% sets name #1 to the expansion of #2
  {\PT@expanded{\@namedef{#1}}}
%    \end{macrocode}
% \end{macro}
%
% \begin{macro}{\PT@addoptargtomacro}
% \begin{macro}{\PT@addargtomacro}
%    \begin{macrocode}
\def\PT@addoptargtomacro
  {\PT@add@argtomacro\PT@makeoptarg}
\def\PT@addargtomacro
  {\PT@add@argtomacro\PT@makearg}

\def\PT@add@argtomacro#1#2#3%
  {\PT@expanded{\PT@expanded{\gdef\PT@temp}}{\csname #3\endcsname}%
   #1%
   \PT@expanded{\PT@gaddendmacro{#2}}{\PT@temp}}

\def\PT@makeoptarg%
  {\PT@expanded{\def\PT@temp}{\expandafter[\PT@temp]}}
\def\PT@makearg%
  {\PT@expanded{\def\PT@temp}{\expandafter{\PT@temp}}}
%    \end{macrocode}
% \end{macro}
%
% \begin{macro}{\PT@gobbleoptional}
% Gobbles one optional argument. Ignores spaces.
%    \begin{macrocode}
\newcommand*{\PT@gobbleoptional}[1][]{\ignorespaces}
%    \end{macrocode}
% \end{macro}
% \end{macro}
%
% \begin{macro}{\PT@addmem}
% \begin{macro}{\PT@splitmem}
% \begin{macro}{\PT@addfile}
% \begin{macro}{\PT@splitfile}
% \begin{macro}{\PT@preparereadmem}
% \begin{macro}{\PT@preparereadfile}
% \begin{macro}{\PT@preparewrite}
% \begin{macro}{\PT@finalizefile}
% \begin{macro}{\PT@queuefilename}
% The following macros handle a simple queue of names.
% With |\PT@addmem|, a name is added to the end of the queue. Using
% |\PT@splitmem|, the first element of the queue is bound to another
% command. As a replacement, we also define |\PT@addfile| and
% |\PT@splitfile|, that implement the queue in a file.
%    \begin{macrocode}
\def\PT@addmem#1#2{\PT@gaddendmacro #2{\PT@elt{#1}}}
\def\PT@splitmem#1#2{#1\PT@nil{#2}{#1}}

\def\PT@elt#1#2\PT@nil#3#4{\gdef #3{#1}\gdef #4{#2}}

\def\PT@queuefilename{\jobname.ptb}

% the \empty at the end consumes the newline space
\def\PT@addfile#1#2{%
  \immediate\write #2{\string\def\string\PTtemp{#1}\string\empty}}

\def\PT@splitfile#1#2{%
  \ifeof #1%
    \let #2=\empty
  \else
    \read #1 to#2%
    %\show #2%
    #2% hack, because it essentially ignores #2
    \PT@expanded{\def #2}{\PTtemp}%
    %\show #2%
  \fi}

%\def\strip#1{\def #1{\expandafter\@strip #1\@dummy}}
%\def\@strip#1\@dummy{#1}

\def\PT@preparereadmem#1#2{%
  \global\let #1=#2}

\def\PT@preparewritefile#1{%
  \immediate\openout\PT@out\PT@queuefilename\relax
  \let #1\PT@out}

\def\PT@preparereadfile#1#2{%
  \immediate\closeout\PT@out
  \openin\PT@in\PT@queuefilename\relax
  \let #1\PT@in}

\def\PT@finalizefile{%
  \closein\PT@in}

\disktables
%    \end{macrocode}
% \end{macro}
% \end{macro}
% \end{macro}
% \end{macro}
% \end{macro}
% \end{macro}
% \end{macro}
% \end{macro}
% \end{macro}
%
% \subsection{The environment}
%
% The general idea is to first scan the contents of the environment
% and store them in a token register. In a few test runs, the positions
% of the column borders are determined. After that, the table is
% typeset by producing boxes of appropriate size.
%
% \begin{macro}{\beginpolytable}
% This macro starts the environment. It should, however, not
% be called directly, but rather in a \LaTeX\ environment.
% We initialize the
% token register to the empty string and then start scanning.
%    \begin{macrocode}
\newcommand*{\beginpolytable}%
%    \end{macrocode}
% We save the current enclosing \LaTeX\ environment in
% |\PT@environment|. This will be the |\end| we will be
% looking for, and this will be the environment we manually
% close in the end.
%    \begin{macrocode}
  {\edef\PT@environment{\@currenvir}%
   \begingroup
   % new in v0.7: save counters
   \PT@savecounters
   \PT@toks{}% initialise token register
   \PT@scantoend}
%    \end{macrocode}
% \end{macro}
% \begin{macro}{\endpolytable}
% This is just defined for convenience.
%    \begin{macrocode}
\let\endpolytable=\relax
%    \end{macrocode}
% \end{macro}
% \begin{macro}{\PT@scantoend}
% Whenenver an |\end| is encountered, we check if it really
% ends the current environment. We store the tokens we have
% read. Once we have found the end of the environment, we
% initialize the column queue and column reference,
% |\PT@columnqueue| and |\PT@columnreference|. The new interface
% commands build the queue during the first test run, containing
% the end columns of specific entries of the table. Later, the
% queue is copied to be the reference, and in the typesetting run,
% the information is used to compute the correct box widths.
% We start with an empty queue, and set the reference to |\undefined|,
% because it is not yet needed. TODO: queue and reference must be
% global variables at the moment; try to change that.
%    \begin{macrocode}
\newcommand{\PT@scantoend}% LaTeX check
\long\def\PT@scantoend #1\end #2%
  {\PT@toks\expandafter{\the\PT@toks #1}%
   \def\PT@temp{#2}%
   \ifx\PT@temp\PT@environment
     \global\let\PT@columnqueue     \empty
     \global\let\PT@columnreference \undefined
     \PT@preparewrite\PT@columnqueue
     \expandafter\PT@getwidths
   \else
     \PT@toks\expandafter{\the\PT@toks\end{#2}}%
     \expandafter\PT@scantoend
   \fi}
%    \end{macrocode}
% \end{macro}
% \begin{macro}{\PT@getwidths}
% Here, we make as many test runs as are necessary to
% determine the correct column widths.
%    \begin{macrocode}
\def\PT@getwidths
%    \end{macrocode}
% We let the |\column| command initialize a column
% in the first run.
%    \begin{macrocode}
  {\let\column         \PT@firstrun@column
%    \end{macrocode}
% There is the possibility to save or restore columns.
% This is new in v0.4.
%    \begin{macrocode}
   \let\savecolumns    \PT@savewidths
   \let\restorecolumns \PT@restorewidths
%    \end{macrocode}
% We \emph{always} define a pseudo-column |@begin@|.
% This denotes the begin of a row. We also define a
% pseudo-column |@end@| denoting the end of a row (as
% of v0.8; and I'm not sure if |@begin@| is still
% needed).
%    \begin{macrocode}
   \column{@begin@}{@{}l@{}}%
   \column{@end@}{}%
   \PT@cols=0\relax%
%    \end{macrocode}
% The two other commands that are allowed inside of the
% environment, namely |\fromto| and |\\| are
% initialized. The |\fromto| command may increase the
% current widths of some columns, if necessary, whereas
% |\\| just resets the counter that keeps track
% of the ``current'' column, to |0|. The command
% |\nextline| is an old name for |\\|.
%    \begin{macrocode}
   \let\fromto          \PT@fromto
   \let\PT@processentry \PT@checkwidth
   \let\PT@scanbegin    \PT@scanbeginfree
   \let\\=              \PT@resetcolumn
   \let\nextline        \PT@resetcolumn
   \let\>=              \PT@fromopt
   \let\==              \PT@from
   \let\<=              \PT@toopt
   \global\PT@changedfalse % nothing has changed so far
   \PT@resetcolumn % we are at the beginning of a line
%    \end{macrocode}
% Now we are ready for a test run.
%    \begin{macrocode}
   \the\PT@toks
   \@ifundefined{PT@scanning}%
     {}{\PT@resetcolumn\relax}%
%    \end{macrocode}
% After the first run, we print extra information.
% We use the contents of the macro |\column| to check
% whether we are in the first run, because it will be
% reset below for all other runs to do nothing.
%    \begin{macrocode}
   \ifx\column\PT@otherrun@column
   \else
      % we are in first run, print extra info
      \PT@prelazylist
      \PT@typeout@{\PT@environment: \the\PT@cols\space columns, %
                                    \PT@Print\PT@allcols}%
      \PT@postlazylist
   \fi
%    \end{macrocode}
% The columns are initialised after the first run.
% Therefore, we make sure that the |\column| command
% won't do much in the other runs. Also, saving and
% restoring columns is no longer needed.
%
% Note that redefining just |\column| or just
% |\PT@firstrun@column| is not enough. In the former
% case, on the next trial run, |\PT@getwidths| will
% restore the old definition. In the latter case, if
% there is no additional trial run, the typesetting
% run will be performed with the wrong definition.
% TODO: I think similar issues exist with saving/restoring?
%    \begin{macrocode}
   \let\column             \PT@otherrun@column
   \let\PT@firstrun@column \PT@otherrun@column
   \let\savecolumns        \PT@gobbleoptional
   \let\restorecolumns     \PT@gobbleoptional
   \let\PT@savewidths      \PT@gobbleoptional
   \let\PT@restorewidths   \PT@gobbleoptional
%    \end{macrocode}
% New in v0.7.1: restore counters after each trial run.
%    \begin{macrocode}
   \PT@restorecounters
%    \end{macrocode}
% If some column widths have indeed changed in the
% test run, this will be indicated by the flag
% |\ifPT@changed|. Depending on this flag, we will
% either loop and rerun, or we will continue in
% |\PT@sortcols|.
%    \begin{macrocode}
   \ifPT@changed
      % we need to rerun if something has changed
      \PT@typeout@{There were changes; another trial run needed.}%
      \expandafter\PT@getwidths
   \else
      % we are done and can do the sorting
      \PT@typeout@{There were no changes; reached fixpoint.}%
      \expandafter\PT@sortcols
   \fi}
%    \end{macrocode}
% \end{macro}
%
% \begin{macro}{\PT@savecounters}
% Save all \LaTeX\ counters so that they can be restored after a
% trial run.
%    \begin{macrocode}
\def\PT@savecounters
  {\begingroup
      \def\@elt ##1%
        {\global\csname c@##1\endcsname\the\csname c@##1\endcsname}%
      \xdef\PT@restorecounters{\cl@@ckpt}%
   \endgroup}
%    \end{macrocode}
% \end{macro}
%
% \begin{macro}{\PT@sortcols}
% The column borders are sorted by their horizontal
% position on the page (width). The they get numbered
% consecutively. After that, we are well prepared to
% typeset the table.
%    \begin{macrocode}
\def\PT@sortcols
%    \end{macrocode}
% First, we sort the list. To make sure that the
% computation is only executed once, we save the
% sorted list by means of an |\edef|. Sorting happens
% with \textsf{lazylist}'s |\Insertsort| which expects
% an order and a list. As order, we provide |\PT@ltwidth|,
% which compares the widths of the columns.
% To prevent expansion of the list structure, given by
% |\Cons| and |\Nil|, we fold the list with the
% |\noexpand|ed versions of the list constructors.
%    \begin{macrocode}
  {\PT@prelazylist
   \edef\PT@sortedlist
     {\Foldr{\noexpand\Cons}{\noexpand\Nil}%
        {\Insertsort\PT@ltmax\PT@allcols}}%
   \PT@typeout@{Sorted columns:}%
   \PT@PrintWidth\PT@sortedlist
   \PT@postlazylist
%    \end{macrocode}
% Now, each column is assigned a number, starting from
% zero.
%    \begin{macrocode}
   \PT@cols=0\relax%
   \PT@prelazylist
   \PT@Execute{\Map\PT@numbercol\PT@sortedlist}%
   \PT@postlazylist
   \edef\PT@lastcol@{\PT@StripColumn\PT@lastcol}%
   \PT@typeout@{Numbered successfully, %
                last column is \PT@lastcol@}%
%    \end{macrocode}
% Now is a good time to save table information, if needed later.
% We will also compare our computed information with the restored
% maximum widths.
%    \begin{macrocode}
      \ifx\PT@currentwidths\empty
      \else
         \PT@typeout@{Saving table information for \PT@currentwidths .}%
         \PT@expanded\PT@saveinformation\PT@currentwidths
      \fi
%    \end{macrocode}
% Finally, we can typeset the table.
%    \begin{macrocode}
   \PT@typeset}
%    \end{macrocode}
% \end{macro}
% \begin{macro}{\PT@typeset}
% We redefine |\fromto| and |\\| to their
% final meaning in the typesetting process. The
% |\fromto| statements will be replaced by appropriate
% calls to |\multicolumn|, whereas the |\\| will
% again reset the counter for the current column, but
% also call the table environment's newline macro.
% Again, |\nextline| is allowed as an old name for |\\|.
%    \begin{macrocode}
\def\PT@typeset
  {\PT@typeout@{Typesetting the table ...}%
   \let\PT@processentry           \PT@placeinbox
   \let\PT@scanbegin              \PT@scanbeginwidth
   \let\\=                        \PT@resetandcr
   \let\nextline                  \PT@resetandcr
   \PT@prepareread\PT@columnreference\PT@columnqueue
%    \end{macrocode}
% The environment that will be opened soon, can, if
% if happens to be |tabular| or |array|, redefines |\\|
% once more, and will redefine it to |\@arraycr|. To
% prevent this, we also set |\@arraycr| to our command.
%    \begin{macrocode}
   \let\@arraycr                  \PT@resetandcr
%    \end{macrocode}
% The array environments keep each line in a group; therefore
% |\PT@resetcolumn|, when executed during the linebreak, will
% not affect the current column counters. By having
% |\PT@resetcolumn| before entering the environment, we ensure
% that the group reset will have the correct effect anyway.
%    \begin{macrocode}
   \PT@resetcolumn % we are at the beginning of a line
%    \end{macrocode}
% Now we start the tabular environment with the
% computed preamble. We redefine the |\\| to whatever
% the environment dictates.
%    \begin{macrocode}
   \PT@begin%
%    \end{macrocode}
% Run, and this time, typeset, the contents.
%    \begin{macrocode}
   \the\PT@toks
   \PT@fill% new in 0.7.3: balance the last line
%    \end{macrocode}
% End the array, close the group, close the environment.
% We are done!
%    \begin{macrocode}
   \PT@finalize% finalize the queue (possibly close file)
   \PT@end
   \endgroup
   \PT@typeout@{Finished.}%
   \expandafter\end\expandafter{\PT@environment}}%
%    \end{macrocode}
% \end{macro}
%
% \subsection{New interface}
%
% From v0.8 on, we offer a more convenient user interface
% for the programming of the columns.
%
% \begin{macro}{\PT@from}
% \begin{macro}{\PT@fromopt}
% \begin{macro}{\PT@toopt}
% The macro |\PT@from| is bound to |\=| in a polytable environment,
% and used to move to the named column specified by its argument. The
% previous column is closed. The variant |\PT@fromopt| is bound to
% |\>| and takes an optional argument instead of a mandatory, which defaults
% to the current column name followed by a single dot |.|. We use an
% empty default which is then redefined to make it easier for the
% user to use the default definition (TODO: explain better).
% Otherwise, it is like |\PT@from|.
%
% The macro |\PT@toopt| is bound to |\<|. Where |\PT@from| ends an
% entry if one is active, and starts a new one, the |\PT@toopt|
% variant only ends the currently active entry.
%    \begin{macrocode}
\newcommand{\PT@from}[1]%
  {\PT@checkendentry{#1}\PT@dofrom{#1}}

\newcommand{\PT@fromopt}[1][]%
  {\def\PT@temp{#1}%
   \ifx\PT@temp\empty
     % set default column name
     \def\PT@temp{\PT@currentcolumn .}%
   \fi
   \PT@expanded\PT@from\PT@temp}

\newcommand{\PT@toopt}[1][]%
  {\def\PT@temp{#1}%
   \ifx\PT@temp\empty
     % set default column name
     \def\PT@temp{\PT@currentcolumn .}%
   \fi
   \PT@expanded\PT@checkendentry\PT@temp
   \let\PT@scanning\undefined}
%    \end{macrocode}
% \end{macro}
% \end{macro}
% \end{macro}
%
% \begin{macro}{\PT@dofrom}
%    \begin{macrocode}
\newcommand*{\PT@dofrom}[1]%
  {\edef\PT@currentcolumn{#1}%
   \let\PT@scanning\PT@currentcolumn
   \let\PT@currentpreamble\relax% necessary for preparescan
   \@ifnextchar[%]
     {\PT@expanded\PT@dospecfrom\PT@currentcolumn}%
     {\PT@expanded\PT@dodofrom  \PT@currentcolumn}}

\newcommand*{\PT@dospecfrom}{}% LaTeX check
\def\PT@dospecfrom #1[#2]%
  {\PT@checkglobalfrom #2\PT@nil{#1}%
   \PT@dodofrom{#1}}

\newcommand*{\PT@checkglobalfrom}{}% LaTeX check
\def\PT@checkglobalfrom
  {\@ifnextchar!\PT@getglobalfrom\PT@ignorefrom}

\newcommand*{\PT@getglobalfrom}{}% LaTeX check
\def\PT@getglobalfrom!#1\PT@nil#2%
  {\column{#2}{#1}}

\newcommand*{\PT@ignorefrom}{}% LaTeX check
\def\PT@ignorefrom #1\PT@nil#2%
  {\def\PT@currentpreamble{#1}}

\newcommand*{\PT@dodofrom}[1]%
  {\@ifundefined{PT@columnreference}%
     {% trial run
      \ifx\column\PT@otherruncolumn
      \else
        % first run
        \let\PT@storeendcolumn\PT@add
      \fi
      \def\PT@temp{@end@}}%
     {% final run
      \PT@split\PT@columnreference\PT@temp
      %\PT@typeout@{splitted: \PT@temp}
     }%
   \PT@expanded{\PT@expanded\PT@preparescan\PT@currentcolumn}\PT@temp
   \PT@scanbegin}

\let\PT@storeendcolumn\@gobbletwo
%    \end{macrocode}
% Here, |\PT@scanbegin| will scan free or using the width, depending
% on the run we are in.
% \end{macro}
%
% \begin{macro}{\PT@fromto}
% This is the implementation for the old explicit |\fromto|
% macro. It takes the start and end columns, and the contents.
% It can be used for all runs.
%    \begin{macrocode}
\newcommand*{\PT@fromto}[3]%
%    \end{macrocode}
% We allow a |\fromto| to follow a new-style command, but we reset
% the current column to undefined, so no text may immediately
% follow a |\fromto| command.
%    \begin{macrocode}
  {\PT@checkendentry{#1}%
   \let\PT@scanning\undefined
%    \end{macrocode}
% We check a switch to prevent nested |\fromto|s.
%    \begin{macrocode}
   \PT@infromto
   \def\PT@infromto{%
     \PackageError{polytable}{Nested fromto}{}}%
%    \end{macrocode}
% Here, the real work is done:
%    \begin{macrocode}
   \let\PT@currentpreamble\relax% necessary for preparescan
   \PT@preparescan{#1}{#2}%
   \PT@scanbegin #3\PT@scanend% defines \@curfield
   \PT@processentry{#1}{#2}%
%    \end{macrocode}
% The commands |\PT@scanbegin| and |\PT@processentry| will perform
% different tasks depending on which run we are in.
%
% We ignore spaces after the |\fromto| command.
%    \begin{macrocode}
   \let\PT@infromto\empty
   \ignorespaces}
%    \end{macrocode}
% \end{macro}
%
% \begin{macro}{\PT@checkendentry}
% This macro checks if we are currently scanning an entry. If so
% (this is detected by checking if |\PT@scanning| is defined),
% we close the entry and handle it (in |\PT@endentry|)
% before returning. The argument
% is the name of the column from which this macro is called.
%    \begin{macrocode}
\newcommand*{\PT@checkendentry}% takes one argument
  {\@ifundefined{PT@scanning}%
     {\let\PT@temp\@gobble}%
     {\let\PT@temp\PT@endentry}%
   \PT@temp}
%\newcommand*{\PT@checkendentry}% takes one argument
%  {\@ifundefined{PT@post@preamble}%
%     {\let\PT@temp\PT@discardentry}%
%     {\let\PT@temp\PT@endentry}%
%   \PT@temp}

%\newcommand*{\PT@discardentry}[1]%
%  {\let\PT@postpreamble=\empty\PT@scanend}

\newcommand*{\PT@endentry}[1]%
  {\PT@scanend
   \edef\PT@temp{#1}%
   \PT@expanded\PT@storeendcolumn\PT@temp\PT@columnqueue
   \let\PT@storeendcolumn\@gobbletwo
   \PT@expanded{\PT@expanded\PT@processentry\PT@currentcolumn}\PT@temp}
%    \end{macrocode}
% \end{macro}
%
% \subsection{The trial runs}
%
% For each column, we store information in macros that are
% based on the column name. We store a column's type (i.e., its
% contribution to the table's preamble), its current width
% (i.e., its the horizontal position where the column will start
% on the page), and later, its number, which will be used for
% the |\multicolumn| calculations.
%
% \begin{macro}{\PT@firstrun@column}
% During the first trial run, we initialise all the columns. We
% store their type, as declared in the |\column| command inside
% the environment, and we set their initial width to 0pt.
% Furthermore, we add the column to the list of all available
% columns, increase the column counter, and tell \TeX\ to ignore
% spaces that might follow the |\column| command.
% New in v0.4.1: We make a case distinction on an empty type
% field to prevent warnings for columns that have been defined
% via |\PT@setmaxwidth| -- see there for additional comments.
% New in v0.4.2: We allow redefinition of width if explicitly
% specified, i.e., not equal to 0pt.
%    \begin{macrocode}
\newcommand\PT@firstrun@column[3][0pt]%
  {\@ifundefined{PT@col@#2.type}%
      {\PT@typeout@{Defining column \PT@aligncol{#2} at #1.}%
       \@namedef{PT@col@#2.type}{#3}%
       \@namedef{PT@col@#2.width}{#1}% initialize the width of the column
       % add the new column to the (sortable) list of all columns
       \PT@consmacro\PT@allcols{PT@col@#2}%
       \advance\PT@cols by 1\relax}%
      {\expandafter\ifx\csname PT@col@#2.type\endcsname\empty
         \relax % will be defined in a later table of the same set
       \else
         \begingroup
         \def\PT@temp{PT@col@#2}%
         \ifx\PT@temp\PT@endcol
           \relax % end column is always redefined
         \else
           \PT@warning{Redefining column #2}%
         \fi
         \endgroup
       \fi
       \@namedef{PT@col@#2.type}{#3}%
       \expandafter\ifdim#1>0pt\relax
         \PT@typeout@{Redefining column #2 at #1.}%
         \@namedef{PT@col@#2.width}{#1}%
       \fi
      }%
%    \end{macrocode}
% For the case that we are saving and there is not yet information
% from the |.aux| file, we define the |.max| and |.trusted| fields
% if they are undefined. If information becomes available later, it
% will overwrite these definitions.
%    \begin{macrocode}
   \@ifundefined{PT@col@#2.max}%
      {\@namedef{PT@col@#2.max}{#1}%
       \expandafter\let\csname PT@col@#2.trusted\endcsname\PT@true}{}%
   \ignorespaces}
%    \end{macrocode}
% \end{macro}
% \begin{macro}{\PT@otherrun@column}
% In all but the first trial run, we do not need any additional
% information about the columns any more, so we just gobble the
% two arguments, but still ignore spaces.
%    \begin{macrocode}
\newcommand\PT@otherrun@column[3][]%
  {\ignorespaces}
%    \end{macrocode}
% \end{macro}
% \begin{macro}{\PT@checkcoldefined}
% This macro verifies that a certain column is defined and
% produces an error message if it is not. New in v0.8: this
% macro implicitly defines the column if we have a default
% column specifier.
%    \begin{macrocode}
\def\PT@checkcoldefined #1%
  {\@ifundefined{PT@col@#1.type}%
      {\@ifundefined{PT@defaultcolumnspec}%
          {\PackageError{polytable}{Undefined column #1}{}}
          {\PT@debug@{Implicitly defining column #1}%
           \PT@expanded{\column{#1}}{\PT@defaultcolumnspec}}}{}%
%    \end{macrocode}
% We also have to define columns with empty specifiers. This
% situation can occur if save/restoring columns that are defined
% by default specifiers.
%    \begin{macrocode}
   \expandafter\ifx\csname PT@col@#1.type\endcsname\empty\relax
      \@ifundefined{PT@defaultcolumnspec}{}%
      {\PT@debug@{Implicitly defining column #1}%
       \PT@expanded{\column{#1}}{\PT@defaultcolumnspec}}%
   \fi}
%    \end{macrocode}
% \end{macro}
% \begin{macro}{\PT@checkwidth}
% Most of the work during the trial runs is done here. We increase
% the widths of certain columns, if necessary. Note that there
% are two conditions that have to hold if |\fromto{A}{B}| is encountered:
% \begin{itemize}
% \item the width of |A| has to be at least the width of the current
%    (i.e., previous) column.
% \item the width of |B| has to be at least the width of |A|, plus the
%    width of the entry.
% \end{itemize}
%    \begin{macrocode}
\def\PT@checkwidth #1#2%
  {\PT@checkcoldefined{#2}% first column should have been checked before
%    \end{macrocode}
% Here, we check the first condition.
%    \begin{macrocode}
   \def\PT@temp{PT@col@#1}%
   \ifx\PT@currentcol\PT@temp
     \PT@debug@{No need to skip columns.}%
   \else
     \PT@colwidth=\expandafter\@nameuse\expandafter
                    {\PT@currentcol.width}\relax
     \ifdim\PT@colwidth>\csname PT@col@#1.width\endcsname\relax
       % we need to change the width
       \PT@debug@{s \PT@aligncol{#1}: %
                  old=\expandafter\expandafter\expandafter
                        \PT@aligndim\csname PT@col@#1.width\endcsname\@@%
                  new=\expandafter\PT@aligndim\the\PT@colwidth\@@}%
       \PT@changedtrue
       \PT@enamedef{PT@col@#1.width}{\the\PT@colwidth}%
     \fi
%    \end{macrocode}
% The same for the untrusted |.max| values.
%    \begin{macrocode}
     \PT@colwidth=\expandafter\@nameuse\expandafter
                    {\PT@currentcol.max}\relax
     \ifdim\PT@colwidth>\csname PT@col@#1.max\endcsname\relax
       % we need to change the width
       \PT@debug@{S \PT@aligncol{#1}: %
                  old=\expandafter\expandafter\expandafter
                        \PT@aligndim\csname PT@col@#1.max\endcsname\@@%
                  new=\expandafter\PT@aligndim\the\PT@colwidth\@@}%
       \PT@changedtrue
       \PT@checkrerun
       \PT@enamedef{PT@col@#1.max}{\the\PT@colwidth}%
     \fi
     \ifnum\csname PT@col@#1.trusted\endcsname=\PT@false\relax
       \ifdim\PT@colwidth=\csname PT@col@#1.max\endcsname\relax
         \PT@debug@{#1=\the\PT@colwidth\space is now trusted}%
         \expandafter\let\csname PT@col@#1.trusted\endcsname\PT@true%
       \fi
     \fi
   \fi
%    \end{macrocode}
% We assume that the current field is typeset in |\@curfield|; we
% can thus measure the width of the box and then test the second condition.
%    \begin{macrocode}
   \PT@expanded{\def\PT@temp}{\the\wd\@curfield}%
   \global\PT@colwidth=\@nameuse{PT@col@#1.width}%
   \global\advance\PT@colwidth by \PT@temp\relax%
   \ifdim\PT@colwidth>\csname PT@col@#2.width\endcsname\relax
     % we need to change the width
     \PT@debug@{#2 (width \PT@temp) starts after #1 (at \csname PT@col@#1.width\endcsname)}%
     \PT@debug@{c \PT@aligncol{#2}: %
                old=\expandafter\expandafter\expandafter
                      \PT@aligndim\csname PT@col@#2.width\endcsname\@@%
                new=\expandafter\PT@aligndim\the\PT@colwidth\@@}%
     \PT@changedtrue
     \PT@enamedef{PT@col@#2.width}{\the\PT@colwidth}%
   \fi
%    \end{macrocode}
% And again, we have to do the same for the untrusted maximums.
%    \begin{macrocode}
   \global\PT@colwidth=\@nameuse{PT@col@#1.max}%
   \global\advance\PT@colwidth by \PT@temp\relax%
   \ifdim\PT@colwidth>\csname PT@col@#2.max\endcsname\relax
     % we need to change the width
     \PT@debug@{C \PT@aligncol{#2}: %
                old=\expandafter\expandafter\expandafter
                      \PT@aligndim\csname PT@col@#2.max\endcsname\@@%
                new=\expandafter\PT@aligndim\the\PT@colwidth\@@}%
     \PT@changedtrue
     \PT@checkrerun
     \PT@enamedef{PT@col@#2.max}{\the\PT@colwidth}%
   \fi
   \ifnum\csname PT@col@#2.trusted\endcsname=\PT@false\relax
     \ifdim\PT@colwidth=\csname PT@col@#2.max\endcsname\relax
       \PT@debug@{#2=\the\PT@colwidth\space is now trusted}%
       \expandafter\let\csname PT@col@#2.trusted\endcsname\PT@true%
     \fi
   \fi
%    \end{macrocode}
% Finally, we update the current column to |#2|.
%    \begin{macrocode}
   \def\PT@currentcol{PT@col@#2}}
%    \end{macrocode}
% \end{macro}
% \begin{macro}{\PT@checkrerun}
% If we have changed something with the trusted widths, we have
% to check whether we are in a situation where we are using previously
% defined columns. If so, we have to rerun \LaTeX.
%    \begin{macrocode}
\def\PT@checkrerun
  {\ifnum\PT@inrestore=\PT@true\relax
     \PT@rerun
   \fi}
%    \end{macrocode}
% \end{macro}
% \begin{macro}{\PT@resetcolumn}
% If the end of a line is encountered, we stop scanning the current
% entry, and reset the current column.
%    \begin{macrocode}
\newcommand*{\PT@resetcolumn}[1][]%
  {\PT@checkendentry{@end@}%
   \let\PT@currentcolumn\empty%
   \let\PT@scanning\undefined
   \let\PT@currentcol\PT@nullcol
   % TODO: remove these lines if they don't work
   %\let\PT@pre@preamble\empty
   %\PT@scanbeginfree
  }
%    \end{macrocode}
% \end{macro}
% \begin{macro}{\PT@nullcol}
% The name of the |@begin@| column as a macro, to be able to compare
% to it with |\ifx|; dito for the |@end@| column.
%    \begin{macrocode}
\def\PT@nullcol{PT@col@@begin@}
\def\PT@endcol{PT@col@@end@}
%    \end{macrocode}
% \end{macro}
%
% \subsection{Sorting and numbering the columns}
%
% Not much needs to be done here, all the work is done by the
% macros supplied by the \textsf{lazylist} package.
% We just provide a few additional commands to facilitate
% their use.
%
% \begin{macro}{\PT@Execute}
% \begin{macro}{\PT@Sequence}
% With |\PT@Execute|, a list of commands (with sideeffects) can
% be executed in sequence. Usually, first a command will be
% mapped over a list, and then the resulting list will be
% executed.
%    \begin{macrocode}
\def\PT@Execute{\Foldr\PT@Sequence\empty}
\def\PT@Sequence #1#2{#1#2}
%    \end{macrocode}
% \end{macro}
% \end{macro}
% \begin{macro}{\PT@ShowColumn}
% This is a debugging macro, that is used to output the list
% of columns in a pretty way. The columns internally get
% prefixes to their names, to prevent name conflicts with
% normal commands. In the debug output, we gobble this prefix
% again.
%    \begin{macrocode}
\def\PT@ShowColumn #1#2%
  {\PT@ShowColumn@{#1}#2\PT@ShowColumn@}
\def\PT@ShowColumn@ #1PT@col@#2\PT@ShowColumn@
  {#1{#2} }
\def\PT@ShowColumnWidth #1%
  {\PT@typeout@{%
     \PT@ShowColumn\PT@aligncol{#1}: 
     \expandafter\expandafter\expandafter
       \PT@aligndim\csname #1.max\endcsname\@@}}
\def\PT@StripColumn #1%
  {\expandafter\PT@StripColumn@#1\PT@StripColumn@}
\def\PT@StripColumn@ PT@col@#1\PT@StripColumn@
  {#1}
%    \end{macrocode}
% \end{macro}
% \begin{macro}{\PT@Print}
% Prints a list of columns, using |\PT@ShowColumn|.
%    \begin{macrocode}
\def\PT@Print#1{\PT@Execute{\Map{\PT@ShowColumn\Identity}#1}}
\def\PT@PrintWidth#1{\PT@Execute{\Map\PT@ShowColumnWidth#1}}
%    \end{macrocode}
% \end{macro}
%
% \begin{macro}{\PT@TeXif}
% This is an improved version of \textsf{lazylist}'s |\TeXif|.
% It does have an additional |\relax| to terminate the condition.
% The |\relax| is gobbled again to keep it fully expandable.
%    \begin{macrocode}
\def\PT@TeXif #1%
  {\expandafter\@gobble#1\relax
     \PT@gobblefalse
   \else\relax
     \gobbletrue
   \fi}
\def\PT@gobblefalse\else\relax\gobbletrue\fi #1#2%
  {\fi #1}
%    \end{macrocode}
% \end{macro}
%
% \begin{macro}{\PT@ltmax}
% The order by which the columns are sorted is given by
% the order on their (untrusted) widths.
%    \begin{macrocode}
\def\PT@ltmax #1#2%
  {\Not{\PT@TeXif{\ifdim\csname #1.max\endcsname>\csname #2.max\endcsname}}}
%    \end{macrocode}
% \end{macro}
%
% \begin{macro}{\PT@numbercol}
% This assigns the next consecutive number to a column.
% We also reassign |PT@lastcol| to remember the final column.
%    \begin{macrocode}
\def\PT@numbercol #1%
  {%\PT@typeout@{numbering #1 as \the\PT@cols}%
   \PT@enamedef{#1.num}{\the\PT@cols}%
   \def\PT@lastcol{#1}%
   \advance\PT@cols by 1\relax}
%    \end{macrocode}
% \end{macro}
%
% \subsection{Typesetting the table}
%
% Remember that there are three important macros that
% occur in the body of the polytable: |\column|, |\fromto|,
% and |\\|. The |\column| macro is only really used
% in the very first trial run, so there is nothing new
% we have to do here, but the other two have to be
% redefined.
%
% \begin{macro}{\PT@resetandcr}
% This is what |\\| does in the typesetting phase.
% It resets the current column, but it also calls the surrounding
% environment's newline macro |\PT@cr| \dots
% If we are \emph{not} in the last column, we insert
% an implicit |fromto|. This is needed for the boxed environment to
% make each column equally wide. Otherwise, if the boxed environment
% is typeset in a centered way, things will go wrong.
%    \begin{macrocode}
\newcommand{\PT@resetandcr}%
  {\PT@expanded\PT@checkendentry\PT@lastcol@%
   \ifx\PT@currentcol\PT@lastcol
   \else
     \ifx\PT@currentcol\PT@nullcol
       \edef\PT@currentcol{\Head{\Tail\PT@sortedlist}}%
     \fi
     \edef\PT@currentcol@{\PT@StripColumn\PT@currentcol}%
     \PT@typeout@{adding implicit fromto at eol from \PT@currentcol@
                                         \space to \PT@lastcol@}%
     \PT@expanded{\PT@expanded\fromto\PT@currentcol@}\PT@lastcol@
   \fi
   \PT@typeout@{Next line ...}%
   \let\PT@scanning\undefined% needed for resetcolumn
   \PT@resetcolumn\PT@cr}
%    \end{macrocode}
% \end{macro}
%
% \begin{macro}{\PT@fill}
% This variant of |\PT@resetandcr| is used at the end of
% the environment, to insert a blank box for the |pboxed|
% environment to balance the widths of all lines. It does
% not start a new line, and does nothing if the current
% column is |@begin@|. TODO: extract commonalities with
% |\PT@resetandcr| into a separate macro.
%    \begin{macrocode}
\newcommand{\PT@fill}%
  {\PT@expanded\PT@checkendentry\PT@lastcol@%
   \ifx\PT@currentcol\PT@lastcol
   \else
     \ifx\PT@currentcol\PT@nullcol
     \else
       \edef\PT@currentcol@{\PT@StripColumn\PT@currentcol}%
       \PT@typeout@{adding implicit fromto from \PT@currentcol@
                                    \space to \PT@lastcol@}%
     \PT@expanded{\PT@expanded\fromto\PT@currentcol@}\PT@lastcol@
   \fi\fi}
%    \end{macrocode}
% \end{macro}
%
% \begin{macro}{\PT@placeinbox}
% This macro is the final-run replacement for |\PT@checkwidth|. We use
% the precomputed width information to typeset the contents of the
% table in aligned boxes. The arguments are the same as for
% |\PT@checkwidth|, i.e., the start and the end columns, and the assumption
% that the entry is contained in the box |\@curfield|.
%    \begin{macrocode}
\def\PT@placeinbox#1#2%
%    \end{macrocode}
% We start by computing the amount of whitespace that
% must be inserted before the entry begins. We then insert that amount
% of space.
%    \begin{macrocode}
  {\PT@colwidth=\@nameuse{PT@col@#1.max}%
   \advance\PT@colwidth by -\expandafter\csname\PT@currentcol.max\endcsname
   \leavevmode
   \edef\PT@temp{\PT@StripColumn\PT@currentcol}%
   \PT@typeout@{adding space of width %
                \expandafter\PT@aligndim\the\PT@colwidth\@@
                (\expandafter\PT@aligncol\expandafter{\PT@temp} %
                -> \PT@aligncol{#1})}%
   \hb@xt@\PT@colwidth{%
     {\@mkpream{@{}l@{}}\@addtopreamble\@empty}%
     \let\CT@row@color\relax% colortbl compatibility
     \let\@sharp\empty%
     %\show\@preamble
     \@preamble}%
%    \end{macrocode}
% The important part is to use the pre-typeset box |\@curfield|.
% This produces real output!
%    \begin{macrocode}
   \PT@typeout@{adding box \space\space of width %
                \expandafter\PT@aligndim\the\wd\@curfield\@@
                (\PT@aligncol{#1} -> \PT@aligncol{#2})}%
   \box\@curfield
%    \end{macrocode}
% Finally, we have to reset the current column and
% ignore spaces.
%    \begin{macrocode}
   \def\PT@currentcol{PT@col@#2}%
   \ignorespaces}%
%    \end{macrocode}
% \end{macro}
%
% \begin{macro}{\PT@preparescan}
% The macro |\PT@preparescan| sets the two macros |\PT@scanbegin|
% and |\PT@scanend| in such a way that they scan the input between
% those two macros and place it in a box. The width of the box is
% determined from the given column names. The name |@end@| can be
% used as a column name is a free scan (a scan without knowing the
% real end column) is desired. To allow redefinition of the preamble,
% we assume that |\PT@currentpreamble| is defined to |\relax| if
% we want it set normally dugin |\PT@preparescan|.
%    \begin{macrocode}
\def\PT@preparescan#1#2%
% First, we check that both columns are defined. This will
% actually define the columns if implicit column definitions are
% enabled.
%    \begin{macrocode}
  {\PT@checkcoldefined{#1}%
   \PT@checkcoldefined{#2}%
   \PT@colwidth=\@nameuse{PT@col@#2.max}%
   \advance\PT@colwidth by -\@nameuse{PT@col@#1.max}\relax%
   \ifmmode
     \PT@debug@{*math mode*}%
     \let\d@llarbegin=$%$
     \let\d@llarend=$%$
     \let\col@sep=\arraycolsep
   \else
     \PT@debug@{*text mode*}%
     \let\d@llarbegin=\begingroup
     \let\d@llarend=\endgroup
     \let\col@sep=\tabcolsep
   \fi
   \ifx\PT@currentpreamble\relax
     \PT@expanded{\PT@expanded{\def\PT@currentpreamble}}%
                 {\csname PT@col@#1.type\endcsname}%
   \fi
%    \end{macrocode}
% Now, we make a preamble using the macro |\@mkpream| from the
% \textsf{array} package. This macro takes a format string as
% argument, and defines |\@preamble| as a result, where |\@sharp|
% occurs in the positions of the column contents.
% We perform the operation in a group to prevent certain redefinitions
% from escaping. The |\@preamble| is set globally anyway.
%    \begin{macrocode}
   {\PT@expanded\@mkpream\PT@currentpreamble%
    \@addtopreamble\@empty}%
   \let\CT@row@color\relax% colortbl compatibility
%    \end{macrocode}
% We split the preamble at the position of the |\@sharp|, using
% some tricks to make sure that there really is precisely one
% occurrence of |\@sharp| in the resulting preamble code, and
% producing an error otherwise. The splitting defines
% |\PT@pre@preamble| and |\PT@post@preamble|. With those and
% the computed |\PT@colwidth|, the scan is successfully prepared.
%    \begin{macrocode}
   \expandafter\PT@splitpreamble\@preamble\@sharp\PT@nil}
%    \end{macrocode}
%
% We now define the splitting of the preamble.
%    \begin{macrocode}
\def\PT@splitpreamble #1\@sharp #2\PT@nil{%
  \let\@sharp=\relax% needed for the following assignment
  \def\PT@terp{#2}%
  \ifx\PT@terp\empty%
    \PackageError{polytable}{Illegal preamble (no columns)}{}%
  \fi
  \PT@splitsplitpreamble{#1}#2\PT@nil}

\def\PT@splitsplitpreamble #1#2\@sharp #3\PT@nil{%
  \def\PT@temp{#3}%
  \ifx\PT@temp\empty%
  \else
    \PackageError{polytable}{Illegal preamble (multiple columns)}{}%
  \fi
  \def\PT@pre@preamble{#1}%
  \def\PT@post@preamble{#2}}%
%    \end{macrocode}
%
% Finally, we can define the scan environment, which depends on
% all the other macros being defined correctly. The macro
% |\PT@scanbegin| is not defined directly, but will be set
% to |\PT@scanbeginfree| during the trial runs and to
% |\PT@scanbeginwidth| during the final run.
%    \begin{macrocode}
\def\PT@scanbeginwidth
   {\PT@scanbegin@{\hbox to \PT@colwidth}}

\def\PT@scanbeginfree
   {\PT@scanbegin@{\hbox}}

\def\PT@scanbegin@#1%
   {\setbox\@curfield #1%
    \bgroup
    \PT@pre@preamble\strut\ignorespaces}

\def\PT@scanend
   {\PT@post@preamble
    \egroup}
%    \end{macrocode}
% \end{macro}
%
% \subsection{Saving and restoring column widths}
%
% Column width information can be saved under a name and thus
% be reused in other tables. The idea is that the command
% |\savecolumns| can be issued inside a polytable to save the
% current column information, and |\restorecolumns| can be used
% to make that information accessible in a later table. All tables
% using the same information should have the same column widths,
% which means that some information might need to be passed back.
% Therefore, we need to write to an auxiliary file.
%
% Both |\savecolumns| and |\restorecolumns| are mapped to the internal
% commands |\PT@savewidths| and |\PT@restorewidths|. Both take an
% optional argument specifying a name for the column width
% information. Thereby, multiple sets of such information can be
% used simultaneously.
%
% One important thing to consider is that the widths read from the
% auxiliary file must not be trusted. The user may have edited the
% source file before the rerun, and therefore, the values read might
% actually be too large (or too small, but this is less dangerous).
%
% The way we solve this problem is to distinguish two width values
% per column: the trusted width, only using information from the
% current run, and the untrusted width, incorportating information
% from the |.aux| file. An untrusted width can become (conditionally)
% trusted if it is reached in the computation with respect to an
% earlier column. (Conditionally, because its trustworthiness still
% depends on the earlier columns being trustworthy.) In the end, we
% can check whether all untrusted widths are conditionally trusted.
%
% We write the final, the maximum widths, into the auxiliary file.
% We perform the
% write operation when we are sure that a specific set is
% no longer used. This is the case when we save a new set under the
% same name, or at the end of the document. The command
% |\PT@verifywidths| takes care of this procedure.
% This command will also check if a rerun is necessary, and issue
% an appropriate warning if that should be the case.
%
% \begin{macro}{\PT@setmaxwidth}
% First, we need a macro to help us interpreting the contents of
% the |.aux| file. New v0.4.1: We need to define the restored
% columns with the |\column| command, because otherwise we will
% have problems in the case that later occurences of tables in
% the document that belong to the same set, but define additional
% columns. (Rerun warnings appear ad infinitum.)
% In v0.4.2: columns with width 0.0 are now always trusted.
%    \begin{macrocode}
\newcommand*{\PT@setmaxwidth}[3][\PT@false]% #2 column name, #3 maximum width
  {\@namedef{PT@col@#2.max}{#3}%
   \ifdim#3=0pt\relax
     \expandafter\let\csname PT@col@#2.trusted\endcsname=\PT@true%
   \else
     \expandafter\let\csname PT@col@#2.trusted\endcsname=#1%
   \fi
   \column{#2}{}}%
%    \end{macrocode}
% \end{macro}
%
% \begin{macro}{\PT@loadtable}
% Now, we can load table information that has been read from the
% |.aux| file. Note that a |\csname| construct expands to |\relax|
% if undefined.
%    \begin{macrocode}
\def\PT@loadtable#1% #1 table id number
  {%\expandafter\show\csname PT@restore@\romannumeral #1\endcsname
   %\show\column
   \PT@typeout@
     {Calling \expandafter\string
                \csname PT@restore@\romannumeral #1\endcsname.}%
   \let\maxcolumn\PT@setmaxwidth
   %\expandafter\show\csname PT@load@\romannumeral #1\endcsname
   \csname PT@restore@\romannumeral #1\endcsname}
%    \end{macrocode}
% \end{macro}
%
% \begin{macro}{\PT@loadtablebyname}
% Often, we want to access table information by a column width set
% name.
%    \begin{macrocode}
\def\PT@loadtablebyname#1% #1 set name
  {\PT@typeout@{Loading table information for column width set #1.}%
   \PT@loadtable{\csname PT@widths@#1\endcsname}}%
%    \end{macrocode}
% \end{macro}
%
% \begin{macro}{\PT@saveinformation}
% In each table for which the widths get reused (i.e., in all tables
% that use either |\savecolumns| or |\restorecolumns|, we have to store
% all important information for further use.
%    \begin{macrocode}
\def\PT@saveinformation#1% #1 set name
  {\PT@expanded{\def\PT@temp}{\csname PT@widths@#1\endcsname}%
   \PT@expanded{\def\PT@temp}%
               {\csname PT@restore@\romannumeral\PT@temp\endcsname}%
   \expandafter\gdef\PT@temp{}% start empty
   % this is: \PT@Execute{\Map{\PT@savecolumn{\PT@temp}}{\Reverse\PT@allcols}}
   \expandafter\PT@Execute\expandafter{\expandafter
     \Map\expandafter{\expandafter\PT@savecolumn
       \expandafter{\PT@temp}}{\Reverse\PT@allcols}}}
%    \end{macrocode}
% \end{macro}
%
% \begin{macro}{\PT@savecolumn}
% A single column is saved by this macro.
%    \begin{macrocode}
\def\PT@savecolumn#1#2% #1 macro name, #2 column name
  {\PT@typeout@{saving column #2 in \string #1 ...}%
   \def\PT@temp{#2}%
   \ifx\PT@temp\PT@nullcol
     \PT@typeout@{skipping nullcol ...}%
% This was a bug: end column cannot be skipped, because
% it can change.
%   \else\ifx\PT@temp\PT@endcol
%     \PT@typeout@{skipping endcol ...}%
   \else
     \PT@typeout@{max=\csname #2.max\endcsname, %
                  width=\csname #2.width\endcsname, %
                  trusted=\csname #2.trusted\endcsname}%
     % we need the column command in here
     % we could do the same in \column, but then the location of
     % \save / \restore matters ...
     \PT@gaddendmacro{#1}{\maxcolumn}%
     \ifnum\csname #2.trusted\endcsname=\PT@true\relax
       \PT@gaddendmacro{#1}{[\PT@true]}%
     \fi
     \edef\PT@temp{\PT@StripColumn{#2}}%
     \PT@addargtomacro{#1}{PT@temp}%
     \PT@addargtomacro{#1}{#2.max}%
     \PT@gaddendmacro{#1}{\column}%
     \PT@addoptargtomacro{#1}{#2.width}%
     \edef\PT@temp{\PT@StripColumn{#2}}%
     \PT@addargtomacro{#1}{PT@temp}%
     \PT@addargtomacro{#1}{#2.type}%
     %\show#1%
%  \fi
   \fi
  }
%    \end{macrocode}
% \end{macro}
%
% \begin{macro}{\PT@savewidths}
% If we really want to save column width information, then the first
% thing we should worry about is that there might already have been
% a set with the name in question. Therefore, we will call
% |\PT@verifywidths| for that set. In the case that there is no set
% of this name yet, we will schedule the set for verification at the
% end of document.
%    \begin{macrocode}
\newcommand*{\PT@savewidths}[1][default@]
  {\PT@typeout@{Executing \string\savecolumns [#1].}%
   \def\PT@currentwidths{#1}%
   \PT@verifywidths{#1}%
%    \end{macrocode}
% We now reserve a new unique number for this column width set
% by increasing the |\PT@table| counter. We then associate
% the given name (or |default@|) with the counter value and
% restore the widths from the |.aux| file if they are present.
%    \begin{macrocode}
   \global\advance\PT@table by 1\relax
   \expandafter\xdef\csname PT@widths@#1\endcsname
     {\the\PT@table}%
   \PT@loadtable{\PT@table}%
   \ignorespaces}
%    \end{macrocode}
% \end{macro}
%
% \begin{macro}{\PT@restorewidths}
% Restoring information is quite simple. We just load all information
% available.
%    \begin{macrocode}
\newcommand*{\PT@restorewidths}[1][default@]
  {\PT@typeout@{Executing \string\restorecolumns [#1].}%
   \def\PT@currentwidths{#1}%
   \let\PT@inrestore\PT@true
   \PT@loadtablebyname{#1}%
   \ignorespaces}
%    \end{macrocode}
% \end{macro}
%
% \begin{macro}{\PT@comparewidths}
%    \begin{macrocode}
\def\PT@comparewidths#1% #1 full column name
  {\@ifundefined{#1.max}%
     {\PT@typeout@{computed width for #1 is fine ...}}%
     {\ifdim\csname #1.max\endcsname>\csname #1.width\endcsname\relax
        \PT@typeout@{Preferring saved width for \PT@StripColumn{#1}.}%
        \PT@changedtrue
        \PT@colwidth=\@nameuse{#1.max}\relax
        \PT@enamedef{#1.width}{\the\PT@colwidth}%
      \fi}}
%    \end{macrocode}
% \end{macro}
%
% \begin{macro}{\PT@trustedmax}
%    \begin{macrocode}
\def\PT@trustedmax#1%
  {\PT@TeXif{\ifnum\csname #1.trusted\endcsname=\PT@true}}
%    \end{macrocode}
% \end{macro}
% \begin{macro}{\PT@equalwidths}
%    \begin{macrocode}
\def\PT@equalwidths#1% #1 full column name
  {\@ifundefined{#1.max}{}%
     {\ifdim\csname #1.max\endcsname=\csname #1.width\endcsname\relax
        \PT@typeout@{col #1 is okay ...}%
      \else
        \PT@rerun% a rerun is needed
      \fi}}
%    \end{macrocode}
% \end{macro}
%
% \begin{macro}{\PT@verifywidths}
%    \begin{macrocode}
\def\PT@verifywidths#1% #1 column width set name
  {\@ifundefined{PT@widths@#1}%
     {\PT@typeout@{Nothing to verify yet for set #1.}%
      \PT@typeout@{Scheduling set #1 for verification at end of document.}%
      \AtEndDocument{\PT@verifywidths{#1}}}%
     {\PT@typeout@{Verifying column width set #1.}%
      \PT@expanded\PT@verify@widths{\csname PT@widths@#1\endcsname}{#1}}}

\def\PT@verify@widths#1#2% #1 set id number, #2 set name
  {\@ifundefined{PT@restore@\romannumeral #1}{}%
     {\begingroup
        \let\column\PT@firstrun@column
        \PT@cols=0\relax%
        \def\PT@allcols{\Nil}%
        \PT@loadtablebyname{#2}%
        \PT@table=#1\relax
        % nullcolumn is not loaded, therefore:
        \@namedef{\PT@nullcol .width}{0pt}%
        % checking trust
        \PT@prelazylist
        \All{\PT@trustedmax}{\PT@allcols}%
           {\PT@typeout@{All maximum widths can be trusted -- writing .max!}%
            \PT@save@table{.max}}%
           {\PT@typeout@{Untrustworthy maximums widths -- writing .width!}%
            \PT@rerun
            \PT@save@table{.width}}%
        \PT@postlazylist
      \endgroup}%
   \PT@typeout@{Verification for #2 successful.}}
%    \end{macrocode}
% \end{macro}
%
% \begin{macro}{\PT@save@table}
% Here we prepare to write maximum column widths to the
% |.aux| file.
%    \begin{macrocode}
\def\PT@save@table#1%
  {\PT@typeout@{Saving column width information.}%
   \if@filesw
     \PT@prelazylist
     {\immediate\write\@auxout{%
        \gdef\expandafter\noexpand
          \csname PT@restore@\romannumeral\PT@table\endcsname
            {\PT@Execute{\Map{\PT@write@column{#1}}{\Reverse\PT@allcols}}}}}%
     \PT@postlazylist
   \fi}
%    \end{macrocode}
% \end{macro}
%
% \begin{macro}{\PT@write@column}
% We define the column command to write to the file.
%    \begin{macrocode}
\def\PT@write@column #1#2%
  {\noexpand\maxcolumn^^J%
    {\PT@StripColumn{#2}}%
    {\@nameuse{#2#1}}}%
%    \end{macrocode}
% \end{macro}
%
% \subsection{The user environments}
%
% It remains to define the environments to be
% called by the user.
% New in v0.8: we add the environments
% |ptboxed| and |pmboxed| for text-mode and math-mode
% boxed environments. In turn, we remove |ptabular| and
% |parray|, and make the point to their new counterparts.
%    \begin{macrocode}
\def\pboxed{%
   \let\PT@begin       \empty
   \let\PT@end         \empty
%    \end{macrocode}
% The following assignment is a hack. If |pboxed| is called
% from within another |tabular|- or |array|-environment, then
% this sometimes does the right thing.
%    \begin{macrocode}
   \ifx\\\PT@arraycr
     \let\PT@cr        \PT@normalcr
   \else
     \let\PT@cr        \\%
   \fi
   \expandafter\beginpolytable\ignorespaces}

\let\endpboxed\endpolytable

\def\ptboxed{%
   \def\PT@begin       {\tabular{@{}l@{}}}%
   \let\PT@end         \endtabular
   \let\PT@cr          \@arraycr
   \expandafter\beginpolytable\ignorespaces}

\let\endptboxed\endpolytable

\def\pmboxed{%
   \def\PT@begin       {\array{@{}l@{}}}%
   \let\PT@end         \endarray
   \let\PT@cr          \@arraycr
   \expandafter\beginpolytable\ignorespaces}

\let\endpmboxed\endpolytable

\let\ptabular     \ptboxed
\let\endptabular  \endptboxed
\let\parray       \pmboxed
\let\endparray    \endpmboxed

%    \end{macrocode}
%
% That is all.
%    \begin{macrocode}
%</package>
%    \end{macrocode}
%
% \Finale
\endinput
